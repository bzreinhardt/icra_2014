\documentclass[10pt]{article}
\usepackage{amsmath}
\usepackage{amsfonts}
\usepackage{amssymb}
\usepackage{todonotes}
\begin{document}


\listoftodos
\todototoc
\todo{Terms to associate with technical meaning:\\
Traversal surface\\
repel \\
attract\\
planar force\\
planar torque\\
coordinate system}
\section{Abstract}
Can you move along a surface in space without propellant or physical contact? On-orbit servicing requires a robot to operate in close proximity to the surface of a target spacecraft in order to inspect, refuel, or repair it.  Robotic operation at such small distances is difficult in space because spacecraft are fragile and often have poorly known dynamics that are never damped out by the environment. Current actuators for locomotion and grasping during on-orbit servicing require physical contact with the target, propellant, or cooperation from the target.  
%
This paper presents an new actuator - the induction coupler - that generates eddy-current forces between a robotic orbital inspector and the conductive exterior of its target. These forces allow the inspector to crawl along the surface of a target while maintaining a safety gap. Sets of induction couplers can exert control forces and torques in all six rigid-body degrees of freedom by strategically repelling and shearing across the surface of the target. Spinning arrays of permanent magnets produce shearing eddy-current forces and current-oscillating electromagnets produce pushing eddy-current forces. 
%
The induction coupler's ability to generate forces is state dependent: its forces and torques depend both on the relative orientation of the coupler to the surface and the surface's geometry. This paper uses an analytical model of eddy-current forces to simulate the set of manoeuvres necessary to generate the control forces and torques that can move and orient a robotic orbital inspector. Experiments on a low-friction test bed demonstrate a successful implementation of the actuator and verify the manoeuvres.       
\section{Introduction}
On-orbit servicing (OOS) is a valuable but difficult robotic task. \cite{} \todo{cite papers on difficulty of OOS} Just as on earth, large assets like the International Space Station (ISS) or geostationary satellites experience wear and unexpected problems that require inspection, repair, or refuelling. \todo{cite paper about uses of OOS} These tasks are well suited for robots because it is less dangerous and expensive to send an inspection vehicle to geostationary orbit or outside the ISS than a human spacewalker. \todo{Is it worthwhile to mention the DARPA Phoenix project?} 
%
Manoeuvring close to a target is essential to OOS and is a particularly risky proposition on orbit. There are presently three methods for an inspector to manoeuvre close to the surface of its target: it can physically grapple the surface to pull itself along; it can use propellant and thrusters; or it can use cooperative, non-contacting electromagnetic systems installed both on the inspector and the target.\todo{cite different systems.}Grappling has many potential risks in an uncertain, low-friction environment. Propellant is expensive and can damage sensitive targets. Cooperation is infeasible in many situations because most spacecraft did not launch with the necessary subsystems. Spacecraft are not designed to be inspected or repaired by robots.
%
However, most spacecraft including the ISS have exteriors made primarily of aluminum plates, curves, and beams. 
\section{Physics Model}
\subsection{Permanent Magnets}
\subsection{Electromagnets}
\section{Movement Flavors}
\subsection{Planar Rotation}
%Define 'planar torque as torque perpendicular to the plane
\subsection{Planar Translation}
%TODO Diagram
%TODO Pseudocode for algorithm 
\subsection{Perpendicular Translation}
%TODO Diagram
\subsection{Perpendicular Rotation}
%TODO Diagram
\section{Experimental Verification}
\section{Conclusion}
A robotic on-orbit service vehicle can use induction couplers can produce locomotive forces in all six rigid body degrees of freedom near a conductive surface. Rotating permanent-magnet arrays generate forces perpendicular to both their axes of rotation and the conductive surface. Current-oscillating electromagnets generate forces directly away from the surface. A pair of rotating arrays can produce $\hat{x}$ and $\hat{y}$ forces in the plane of the traversal surface. These two arrays can also produce torques in the $\hat{z}$ direction. This pair of arrays can take advantage of the surface geometry to generate forces in the $\hat{z}$ direction by generating shear forces along a non-flat surface whose $\hat{x}$ and $\hat{y}$ components cancel but whose $\hat{s}$ components add. A single electromagnet generates force in the $+\hat{z}$ direction. The force from the electromagnet can also generate torques based on its relative location to the system's center of mass. These maneuvers are demonstrated in simulations and verified on a low-friction testbed.  

%An autonomous space vehicle can use these forces to crawl over the conductive surface of a target spacecraft without propellant or the risk associated with physical grappling. Rotating arrays of magnets produce planar forces and torques. These rotating arrays can also produce forces perpendicular to the surface by taking advantage of surface features such as curvature or edges. Electromagnets can produce force directly away from any surface and by coupling these forces, produce torques parallel to that surface. Dynamic simulations demonstrate each degree-of-freedom using an analytical force model and hardware demonstrations on a low-friction testbed verify the results. 
%TODO future work - trajectory optimization, end effector design, kinematic simulation
Future work will focus on two areas - adaptive controllers and movement planning. A robotic inspector will need adaptive controllers to account for induction coupler's strong dependence on poorly known parameters of the environment.  The inspector will also need to plan movements carefully because its ability to exert control with an induction coupler is based on both its state, the local geometry.      
\end{document}