\documentclass[10pt]{article}
\usepackage{amsmath}
\usepackage{amsfonts}
\usepackage{amssymb}
\begin{document}
\section{Abstract}
\section{Introduction}
\section{Physics Models}
\subsection{Permanent Magnets}
\subsection{Electromagnets}
\section{Movement Flavors}
\subsection{Planar Rotation}
%Define 'planar torque as torque perpendicular to the plane
\subsection{Planar Translation}
%TODO Diagram
%TODO Pseudocode for algorithm 
\subsection{Perpendicular Translation}
%TODO Diagram
\subsection{Perpendicular Rotation}
%TODO Diagram
\section{Experimental Verification}
\section{Conclusion}
Induction couplers can produce locomotive forces in all six rigid body degrees of freedom near a conductive surface. An autonomous space vehicle can use these forces to crawl over the conductive surface of a target spacecraft without propellant or the risk associated with physical grappling. Rotating arrays of magnets produce planar forces and torques. These rotating arrays can also produce forces perpendicular to the surface by taking advantage of surface features such as curvature or edges. Electromagnets can produce force directly away from any surface and by coupling these forces, produce torques parallel to that surface. Dynamic simulations demonstrate each degree-of-freedom using an analytical force model and hardware demonstrations on a low-friction testbed verify the results. 
%TODO future work - trajectory optimization, end effector design, kinematic simulation
Future work will integrate the geometry-dependent forces of induction couplers into higher level trajectory optimization and      
\end{document}